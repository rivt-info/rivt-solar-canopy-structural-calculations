\documentclass[10pt]{article}
\usepackage[utf8]{inputenc}
\usepackage[T1]{fontenc}
\usepackage{amsmath}
\usepackage{amsfonts}
\usepackage{amssymb}
\usepackage[version=4]{mhchem}
\usepackage{stmaryrd}
\usepackage{graphicx}
\usepackage[export]{adjustbox}
\graphicspath{ {./images/} }

\title{CHAPTER 2-NOTATION AND TERMINOLOGY CODE }

\author{}
\date{}


\begin{document}
\maketitle
\section*{2.1-Scope}
2.1.1 This chapter defines notation and terminology used in this Code.

\section*{2.2-Notation}
$a=$ depth of equivalent rectangular stress block, in.

$a_{v}=$ shear span, equal to distance from center of concentrated load to either: (a) face of support for continuous or cantilevered members, or (b) center of support for simply supported members, in.

$A_{b}=$ area of an individual bar or wire, in.  

$A_{b p}=$ area of the attachment base plate in contact with concrete or grout when loaded in compression, in.  

$A_{\text {brg }}=$ net bearing area of the head of stud, anchor bolt, or headed deformed bar, in.  

$A_{c}=$ area of concrete section resisting shear transfer, in.  

$A_{c f}=$ greater gross cross-sectional area of the two orthogonal slab-beam strips intersecting at a column of a two-way prestressed slab, in.  

$A_{c h}=$ cross-sectional area of a member measured to the outside edges of transverse reinforcement, in.  

$A_{c p}=$ area enclosed by outside perimeter of concrete cross section, in.  

$A_{c s}=$ cross-sectional area at one end of a strut in a strutand-tie model, taken perpendicular to the axis of the strut, in.  

$A_{c t}=$ area of that part of cross section between the flexural tension face and centroid of gross section, in.  

$A_{c v}=$ gross area of concrete section bounded by web thickness and length of section in the direction of shear force considered in the case of walls, and gross area of concrete section in the case of diaphragms. Gross area is total area of the defined section minus area of any openings, in.  

$A_{c w}=$ area of concrete section of an individual pier, horizontal wall segment, or coupling beam resisting shear, in.  

$\mid A_{e f, s l}=$ effective bearing area of shear lug, in $^{2}$.

$A_{f}=$ area of reinforcement in bracket or corbel resisting design moment, in.  

$A_{g}=$ gross area of concrete section, in.   For a hollow section, $A_{g}$ is the area of the concrete only and does not include the area of the void(s)

$A_{h}=$ total area of shear reinforcement parallel to primary tension reinforcement in a corbel or bracket, in.  

$A_{h s}=$ total cross-sectional area of hooked or headed bars being developed at a critical section, in.  

$A_{j}=$ effective cross-sectional area within a joint in a plane parallel to plane of beam reinforcement generating shear in the joint, in.  

$A_{\ell}=$ total area of longitudinal reinforcement to resist torsion, in.  

$A_{\ell, \text { min }}=$ minimum area of longitudinal reinforcement to resist torsion, in.  

\section*{CODE}
$A_{n}=$ area of reinforcement in bracket or corbel resisting factored restraint force $N_{u c}$, in.  

$A_{n z}=$ area of a face of a nodal zone or a section through a nodal zone, in.  

$A_{N a}=$ projected influence area of a single adhesive anchor or group of adhesive anchors, for calculation of bond strength in tension, in.  

$A_{N a o}=$ projected influence area of a single adhesive anchor, for calculation of bond strength in tension if not limited by edge distance or spacing, in.  

$A_{N c}=$ projected concrete failure area of a single anchor or group of anchors, for calculation of strength in tension, in.  

$A_{N c o}=$ projected concrete failure area of a single anchor, for calculation of strength in tension if not limited by edge distance or spacing, in.  

$A_{o}=$ gross area enclosed by torsional shear flow path, in. $^{2}$

$A_{o h}=$ area enclosed by centerline of the outermost closed transverse torsional reinforcement, in.  

$A_{p d}=$ total area occupied by duct, sheathing, and prestressing reinforcement, in.  

$A_{p s}=$ area of prestressed longitudinal tension reinforcement, in.  

$A_{p t}=$ total area of prestressing reinforcement, in.  

$A_{s}=$ area of nonprestressed longitudinal tension reinforcement, in.  

$A_{s}{ }^{\prime}=$ area of compression reinforcement, in.  

$A_{s c}=$ area of primary tension reinforcement in a corbel or bracket, in.  

$A_{s e, N}=$ effective cross-sectional area of anchor in tension, in. $^{2}$

$A_{s e, V}=$ effective cross-sectional area of anchor in shear, in. $^{2}$

$A_{s h}=$ total cross-sectional area of transverse reinforcement, including crossties, within spacing $s$ and perpendicular to dimension $b_{c}$, in.  

$A_{s i}=$ total area of surface reinforcement at spacing $s_{i}$ in the $i$-th layer crossing a strut, with reinforcement at an angle $\alpha_{i}$ to the axis of the strut, in.  

$A_{s, \text { min }}=$ minimum area of flexural reinforcement, in.  

$A_{s t}=$ total area of nonprestressed longitudinal reinforcement including bars or steel shapes, and excluding prestressing reinforcement, in.  

$A_{t}=$ area of one leg of a closed stirrup, hoop, or tie resisting torsion within spacing $s$, in.  

$A_{\text {th }}=$ total cross-sectional area of ties or stirrups confining hooked bars, in.  

$A_{t p}=$ area of prestressing reinforcement in a tie, in.  

$A_{t r}=$ total cross-sectional area of all transverse reinforcement within spacing $s$ that crosses the potential plane of splitting through the reinforcement being developed, in.  

$A_{t s}=$ area of nonprestressed reinforcement in a tie, in.  

\section*{COMMENTARY}
\section*{CODE}
$A_{t t}=$ total cross-sectional area of ties or stirrups acting as parallel tie reinforcement for headed bars, in.  

$A_{v}=$ area of shear reinforcement within spacing $s$, in. $^{2}$

$A_{v d}=$ total area of reinforcement in each group of diagonal bars in a diagonally reinforced coupling beam, in.  

$A_{v f}=$ area of shear-friction reinforcement, in.  

$A_{v h}=$ area of shear reinforcement parallel to flexural tension reinforcement within spacing $s_{2}$, in.  

$A_{v, \text { min }}=$ minimum area of shear reinforcement within spacing $s$, in.  

$A_{V c}=$ projected concrete failure area of a single anchor or group of anchors, for calculation of strength in shear, in.  

$A_{V c o}=$ projected concrete failure area of a single anchor, for calculation of strength in shear, if not limited by corner influences, spacing, or member thickness, in.  

$A_{1}=$ loaded area for consideration of bearing, strut, and node strength, in.  

$A_{2}=$ area of the lower base of the largest frustum of a pyramid, cone, or tapered wedge contained wholly within the support and having its upper base equal to the loaded area. The sides of the pyramid, cone, or tapered wedge shall be sloped one vertical to two horizontal, in.  

$b=$ width of compression face of member, in.

$b_{c}=$ cross-sectional dimension of member core measured to the outside edges of the transverse reinforcement composing area $A_{s h}$, in.

$b_{f}=$ effective flange width, in.

$b_{o}=$ perimeter of critical section for two-way shear in slabs and footings, in.

$b_{s}=$ width of strut, in

$\mid b_{s l}=$ width of shear lug, in.

$b_{\text {slab }}=$ effective slab width, in

$b_{t}=$ width of that part of cross section containing the closed stirrups resisting torsion, in.

$b_{v}=$ width of cross section at contact surface being investigated for horizontal shear, in.

$b_{w}=$ web width or diameter of circular section, in.

$b_{1}=$ dimension of the critical section $b_{o}$ measured in the direction of the span for which moments are determined, in.

$b_{2}=$ dimension of the critical section $b_{o}$ measured in the direction perpendicular to $b_{1}$, in.

$B_{n}=$ nominal bearing strength, $\mathrm{lb}$

$B_{u}=$ factored bearing load, $\mathrm{lb}$

c $=$ distance from extreme compression fiber to neutral axis, in.

$c_{a c}=$ critical edge distance required to develop the basic strength as controlled by concrete breakout or bond of a post-installed anchor in tension in uncracked concrete without supplementary reinforcement to control splitting, in.\\
COMMENTARY

\section*{CODE}
$c_{a, \max }=$ maximum distance from center of an anchor shaft to the edge of concrete, in.

$c_{a, \text { min }}=$ minimum distance from center of an anchor shaft to the edge of concrete, in.

$c_{a 1}=$ distance from the center of an anchor shaft to the edge of concrete in one direction, in. If shear is applied to anchor, $c_{a 1}$ is taken in the direction of the applied shear. If tension is applied to the anchor, $c_{a 1}$ is the minimum edge distance. Where anchors subject to shear are located in narrow sections of limited thickness, see R17.7.2.1.2

$c_{a 2}=$ distance from center of an anchor shaft to the edge of concrete in the direction perpendicular to $c_{a}$, in.

$c_{b}=$ lesser of: (a) the distance from center of a bar or wire to nearest concrete surface, and (b) one-half the center-to-center spacing of bars or wires being developed, in.

$c_{c}=$ clear cover of reinforcement, in.

$c_{\mathrm{Na}}=$ projected distance from center of an anchor shaft on one side of the anchor required to develop the full bond strength of a single adhesive anchor, in.

$c_{s l}=$ distance from the centerline of the row of anchors in tension nearest the shear lug to the centerline of the shear lug measured in the direction of shear, in.

$=$ distance from the interior face of the column to the slab edge measured parallel to $c_{1}$, but not exceeding $c_{1}$, in.

$c_{1}=$ dimension of rectangular or equivalent rectangular column, capital, or bracket measured in the direction of the span for which moments are being determined, in.

$c_{2}=$ dimension of rectangular or equivalent rectangular column, capital, or bracket measured in the direction perpendicular to $c_{1}$, in.

$C_{m}=$ factor relating actual moment diagram to an equivalent uniform moment diagram

$d=$ distance from extreme compression fiber to centroid of longitudinal tension reinforcement, in.

$d^{\prime}=$ distance from extreme compression fiber to centroid of longitudinal compression reinforcement, in.

$d_{a}=$ outside diameter of anchor or shaft diameter of headed stud, headed bolt, or hooked bolt, in.

$d_{a}{ }^{\prime}=$ value substituted for $d_{a}$ if an oversized anchor is used, in.

$d_{a g g}=$ nominal maximum size of coarse aggregate, in.

$d_{b}=$ nominal diameter of bar, wire, or prestressing strand, in.

$d_{p} \quad=$ distance from extreme compression fiber to centroid of prestressed reinforcement, in.

\section*{COMMENTARY}
$c_{a 1}^{\prime}=$ limiting value of $c_{a 1}$ where anchors are located less than $1.5 c_{a 1}$ from three or more edges, in.; see Fig. R17.7.2.1.2

$C=$ compressive force acting on a nodal zone, $\mathrm{lb}$

$d_{\text {burst }}=$ distance from the anchorage device to the centroid of the bursting force, $T_{\text {burst }}$, in.

\section*{CODE}
$d_{\text {pile }}=$ diameter of pile at footing base, in.

$D=$ effect of service dead load

$D_{s}=$ effect of superimposed dead load

$D_{w}=$ effect of self-weight dead load of the concrete structural system

$e_{h}=$ distance from the inner surface of the shaft of a $\mathrm{J}-$ or L-bolt to the outer tip of the J- or L-bolt, in.

$e_{N}^{\prime}=$ distance between resultant tension load on a group of anchors loaded in tension and the centroid of the group of anchors loaded in tension, in.; $e_{N}^{\prime}$ is always positive

$e_{V}^{\prime}=$ distance between resultant shear load on a group of anchors loaded in shear in the same direction, and the centroid of the group of anchors loaded in shear in the same direction, in.; $e_{V}^{\prime}$ is always positive

$E=$ effect of horizontal and vertical earthquake-induced forces

$E_{c}=$ modulus of elasticity of concrete, $\mathrm{psi}$

$E_{c b}=$ modulus of elasticity of beam concrete, psi

$E_{c s}=$ modulus of elasticity of slab concrete, psi

$E I=$ flexural stiffness of member, in. ${ }^{2}-\mathrm{lb}$

$(E)_{\text {eff }}=$ effective flexural stiffness of member, in. ${ }^{2}-\mathrm{lb}$

$E_{p}=$ modulus of elasticity of prestressing reinforcement, psi

$E_{s}=$ modulus of elasticity of reinforcement and structural steel, excluding prestressing reinforcement, psi

$f_{c}^{\prime}=$ specified compressive strength of concrete, psi

$\sqrt{f_{c}^{\prime}}=$ square root of specified compressive strength of concrete, psi

$f_{c i}^{\prime}=$ specified compressive strength of concrete at time of initial prestress, psi

$\sqrt{f_{c i}^{\prime}}=$ square root of specified compressive strength of concrete at time of initial prestress, psi

$f_{c e}=$ effective compressive strength of the concrete in a strut or a nodal zone, psi

$f_{d}=$ stress due to unfactored dead load, at extreme fiber of section where tensile stress is caused by externally applied loads, psi

$f_{d c}=$ decompression stress; stress in the prestressed reinforcement if stress is zero in the concrete at the same level as the centroid of the prestressed reinforcement, psi

$=$ compressive stress in concrete, after allowance for all prestress losses, at centroid of cross section resisting externally applied loads or at junction of web and flange where the centroid lies within the flange, psi. In a composite member, $f_{p c}$ is the resultant compressive stress at centroid of composite section, or at junction of web and flange where the centroid lies within the flange, due to both prestress

\section*{COMMENTARY}
$e_{\text {anc }}=$ eccentricity of the anchorage device or group of devices with respect to the centroid of the cross section, in.

\section*{CODE}
and moments resisted by precast member acting alone

$f_{p e}=$ compressive stress in concrete due only to effective prestress forces, after allowance for all prestress losses, at extreme fiber of section if tensile stress is caused by externally applied loads, psi

$=$ stress in prestressed reinforcement at nominal flexural strength, psi

$=$ specified tensile strength of prestressing reinforcement, psi

$f_{p y}=$ specified yield strength of prestressing reinforcement, psi

$=$ modulus of rupture of concrete, psi

$=$ tensile stress in reinforcement at service loads, excluding prestressed reinforcement, psi

$=$ compressive stress in reinforcement under factored loads, excluding prestressed reinforcement, psi

$=$ effective stress in prestressed reinforcement, after allowance for all prestress losses, psi

$=$ extreme fiber stress in the precompressed tension zone calculated at service loads using gross section properties after allowance of all prestress losses, psi

$f_{\text {uta }}=$ specified tensile strength of anchor steel, psi

$f_{y}=$ specified yield strength for nonprestressed reinforcement, psi

$f_{y a}=$ specified yield strength of anchor steel, psi

$f_{y t}=$ specified yield strength of transverse reinforcement, psi

$F=$ effect of service load due to fluids with well-defined pressures and maximum heights

$F_{n n}=$ nominal strength at face of a nodal zone, $\mathrm{lb}$

$F_{n s}=$ nominal strength of a strut, $\mathrm{lb}$

$F_{n t}=$ nominal strength of a tie, $1 \mathrm{~b}$

$F_{u n}=$ factored force on the face of a node, $\mathrm{lb}$

$F_{u s}=$ factored compressive force in a strut, $\mathrm{lb}$

$F_{u t}=$ factored tensile force in a tie, $\mathrm{lb}$

$h=$ overall thickness, height, or depth of member, in.

$h_{a}=$ thickness of member in which an anchor is located, measured parallel to anchor axis, in.

$h_{e f}=$ effective embedment depth of anchor, in.

$h_{e f, s l}=$ effective embedment depth of shear lug, in.

$h_{s l}=$ embedment depth of shear lug, in.

$h_{s x}=$ story height for story $x$, in.

$h_{u}=$ laterally unsupported height at extreme compression fiber of wall or wall pier, in., equivalent to $\ell_{u}$ for compression members

\section*{COMMENTARY}
$h_{\text {anc }}=$ dimension of anchorage device or single group of closely spaced devices in the direction of bursting being considered, in.

$h_{e f}^{\prime}=$ limiting value of $h_{e f}$ where anchors are located less than $1.5 h_{e f}$ from three or more edges, in.; refer to Fig. R17.6.2.1.2

\section*{CODE}
$h_{w}=$ height of entire wall from base to top, or clear height of wall segment or wall pier considered, in.

$h_{w c s}=$ height of entire structural wall above the critical section for flexural and axial loads, in.

$h_{x}=$ maximum center-to-center spacing of longitudinal bars laterally supported by corners of crossties or hoop legs around the perimeter of a column or wall boundary element, in.

$H=$ effect of service load due to lateral earth pressure, ground water pressure, or pressure of bulk materials, $\mathrm{lb}$

$I=$ moment of inertia of section about centroidal axis, in. ${ }^{4}$

$I_{b} \quad=$ moment of inertia of gross section of beam about centroidal axis, in. ${ }^{4}$

$I_{c r}=$ moment of inertia of cracked section transformed to concrete, in. ${ }^{4}$

$I_{e}=$ effective moment of inertia for calculation of deflection, in. ${ }^{4}$

$I_{g}=$ moment of inertia of gross concrete section about centroidal axis, neglecting reinforcement, in. ${ }^{4}$

$I_{s}=$ moment of inertia of gross section of slab about centroidal axis, in. ${ }^{4}$

$I_{s e}=$ moment of inertia of reinforcement about centroidal axis of member cross section, in. ${ }^{4}$

$k=$ effective length factor for compression members

$k_{c}=$ coefficient for basic concrete breakout strength in tension

$k_{c p}=$ coefficient for pryout strength

$k_{f}=$ concrete strength factor

$k_{n}=$ confinement effectiveness factor

$K_{t r}=$ transverse reinforcement index, in.

$\ell=$ span length of beam or one-way slab; clear projection of cantilever, in.

$\ell_{b e}=$ length of boundary element from compression face of member, in.

$\ell_{a}=$ additional embedment length beyond centerline of support or point of inflection, in.

$\ell_{c}=$ length of compression member, measured centerto-center of the joints, in.

$\ell_{c b}=$ arc length of bar bend along centerline of bar, in.

$\ell_{d}=$ development length in tension of deformed bar, deformed wire, plain and deformed welded wire reinforcement, or pretensioned strand, in.

$\ell_{d c}=$ development length in compression of deformed bars and deformed wire, in.

$\ell_{d b}=$ debonded length of prestressed reinforcement at end of member, in.

\section*{COMMENTARY}
$K_{t}=$ torsional stiffness of member; moment per unit rotation

$K_{05}=$ coefficient associated with the 5 percent fractile

$\ell_{\text {anc }}=$ length along which anchorage of a tie must occur, in.

$\ell_{b}=$ width of bearing, in.

\section*{CODE}
$\ell_{d h}=$ development length in tension of deformed bar or deformed wire with a standard hook, measured from outside end of hook, point of tangency, toward critical section, in.

$\ell_{d t}=$ development length in tension of headed deformed bar, measured from the bearing face of the head toward the critical section, in.

$\ell_{e}=$ load bearing length of anchor for shear, in.

$\ell_{\text {ext }}=$ straight extension at the end of a standard hook, in.

$\ell_{n}=$ length of clear span measured face-to-face of supports, in.

$\ell_{0}=$ length, measured from joint face along axis of member, over which special transverse reinforcement must be provided, in.

$\ell_{s c}=$ compression lap splice length, in.

$\ell_{s t}=$ tension lap splice length, in.

$\ell_{t}=$ span of member under load test, taken as the shorter span for two-way slab systems, in. Span is the lesser of: (a) distance between centers of supports, and (b) clear distance between supports plus thickness $h$ of member. Span for a cantilever shall be taken as twice the distance from face of support to cantilever end

$\ell_{t r}=$ transfer length of prestressed reinforcement, in.

$\ell_{u}=$ unsupported length of column or wall, in.

$\ell_{w}=$ length of entire wall, or length of wall segment or wall pier considered in direction of shear force, in.

$\ell_{1}=$ length of span in direction that moments are being determined, measured center-to-center of supports, in.

$\ell_{2}=$ length of span in direction perpendicular to $\ell_{1}$, measured center-to-center of supports, in.

$L \quad=\quad$ effect of service live load

$L_{r} \quad=$ effect of service roof live load

$M_{a}=$ maximum moment in member due to service loads at stage deflection is calculated, in.-lb

$M_{c}=$ factored moment amplified for the effects of member curvature used for design of compression member, in.-lb

$M_{c r}=$ cracking moment, in.-lb

$M_{\text {cre }}=$ moment causing flexural cracking at section due to externally applied loads, in.-lb

$M_{\max }=$ maximum factored moment at section due to externally applied loads, in.-lb

$M_{n}=$ nominal flexural strength at section, in.-lb

$M_{n b}=$ nominal flexural strength of beam including slab where in tension, framing into joint, in.-lb

$M_{n c}=$ nominal flexural strength of column framing into joint, calculated for factored axial force, consistent with the direction of lateral forces considered, resulting in lowest flexural strength, in.-1b

$M_{p r}=$ probable flexural strength of members, with or without axial load, determined using the properties of the member at joint faces assuming a tensile

\section*{COMMENTARY}
$M=$ moment acting on anchor or anchor group, in.-lb

\section*{CODE}
stress in the longitudinal bars of at least $1.25 f_{y}$ and a strength reduction factor $\phi$ of 1.0 , in.-lb

$M_{s a}=$ maximum moment in wall due to service loads, excluding $P \Delta$ effects, in.-lb

$M_{s c}=$ factored slab moment that is resisted by the column at a joint, in.-lb

$M_{u}=$ factored moment at section, in. $-1 \mathrm{~b}$

$M_{u a}=$ moment at midheight of wall due to factored lateral and eccentric vertical loads, not including $P \triangle$ effects, in.-lb

$M_{1}=$ lesser factored end moment on a compression member, in.-lb

$M_{1 n s}=$ factored end moment on a compression member at the end at which $M_{1}$ acts, due to loads that cause no appreciable sidesway, calculated using a first-order elastic frame analysis, in.-lb

$M_{1 s}=$ factored end moment on compression member at the end at which $M_{1}$ acts, due to loads that cause appreciable sidesway, calculated using a first-order elastic frame analysis, in.-lb

$M_{2}=$ greater factored end moment on a compression member. If transverse loading occurs between supports, $M_{2}$ is taken as the largest moment occurring in member. Value of $M_{2}$ is always positive, in.-lb

$M_{2, \min }=$ minimum value of $M_{2}$, in.-1b

$M_{2 n s}=$ factored end moment on compression member at the end at which $M_{2}$ acts, due to loads that cause no appreciable sidesway, calculated using a first-order elastic frame analysis, in.-lb

$M_{2 s}=$ factored end moment on compression member at the end at which $M_{2}$ acts, due to loads that cause appreciable sidesway, calculated using a first-order elastic frame analysis, in.-lb

$n \quad=$ number of items, such as, bars, wires, monostrand anchorage devices, or anchors

$=$ number of longitudinal bars around the perimeter of a column core with rectilinear hoops that are laterally supported by the corner of hoops or by seismic hooks. A bundle of bars is counted as a single bar

$\mid n_{s} \quad=$ number of stories above the critical section

$N_{a}=$ nominal bond strength in tension of a single adhesive anchor, $\mathrm{lb}$

$N_{a g}=$ nominal bond strength in tension of a group of adhesive anchors, $\mathrm{lb}$

$N_{b}=$ basic concrete breakout strength in tension of a single anchor in cracked concrete, lb

$N_{b a}=$ basic bond strength in tension of a single adhesive anchor, $\mathrm{lb}$

$N_{c}=$ resultant tensile force acting on the portion of the concrete cross section that is subjected to tensile stresses due to the combined effects of service loads and effective prestress, $\mathrm{lb}$

\section*{COMMENTARY}
\section*{CODE}
\section*{COMMENTARY}
$N_{c b}=$ nominal concrete breakout strength in tension of a single anchor, $\mathrm{lb}$

$N_{c b g}=$ nominal concrete breakout strength in tension of a group of anchors, $\mathrm{lb}$

$N_{c p}=$ basic concrete pryout strength of a single anchor, $\mathrm{lb}$

$N_{c p g}=$ basic concrete pryout strength of a group of anchors, $\mathrm{lb}$

$N_{n}=$ nominal strength in tension, $\mathrm{lb}$

$N_{p}=$ pullout strength in tension of a single anchor in cracked concrete, $\mathrm{lb}$

$N_{p n}=$ nominal pullout strength in tension of a single anchor, $1 \mathrm{~b}$

$N_{s a}=$ nominal strength of a single anchor or individual anchor in a group of anchors in tension as governed by the steel strength, $\mathrm{lb}$

$N_{s b}=$ side-face blowout strength of a single anchor, $\mathrm{lb}$

$N_{s b g}=$ side-face blowout strength of a group of anchors, $1 \mathrm{~b}$

$N_{u}=$ factored axial force normal to cross section occurring simultaneously with $V_{u}$ or $T_{u}$; to be taken as positive for compression and negative for tension, $\mathrm{lb}$

$N_{u a}=$ factored tensile force applied to anchor or individual anchor in a group of anchors, $1 \mathrm{~b}$

$N_{u a, g}=$ total factored tensile force applied to anchor group, $\mathrm{lb}$

$N_{u a, i}=$ factored tensile force applied to most highly stressed anchor in a group of anchors, lb

$N_{u a, s}=$ factored sustained tension load, $\mathrm{lb}$

$N_{u c}=$ factored restraint force applied to a bearing connection acting perpendicular to and simultaneously with $V_{u}$, to be taken as positive for tension, $\mathrm{lb}$

$N_{u c, \text { max }}=$ maximum restraint force that can be transmitted through the load path of a bearing connection multiplied by the load factor used for live loads in combinations with other factored load effects

$p_{c p}=$ outside perimeter of concrete cross section, in.

$p_{h}=$ perimeter of centerline of outermost closed transverse torsional reinforcement, in.

$P_{a}=$ maximum allowable compressive strength of a deep foundation member, $\mathrm{lb}$

$P_{c}=$ critical buckling load, $\mathrm{lb}$

$P_{n}=$ nominal axial compressive strength of member, $\mathrm{lb}$

$P_{n, \max }=$ maximum nominal axial compressive strength of a member, $\mathrm{lb}$

$P_{n t}=$ nominal axial tensile strength of member, $\mathrm{lb}$

$P_{n, \max }=$ maximum nominal axial tensile strength of member, $\mathrm{lb}$

$P_{o}=$ nominal axial strength at zero eccentricity, $\mathrm{lb}$

$P_{p u}=$ factored prestressing force at anchorage device, $\mathrm{lb}$

$P_{s}=$ unfactored axial load at the design, midheight section including effects of self-weight, $\mathrm{lb}$

$P_{u}=$ factored axial force; to be taken as positive for compression and negative for tension, $\mathrm{lb}$

$P \delta=$ secondary moment due to individual member slenderness, in.-lb

\section*{CODE}
$P \Delta=$ secondary moment due to lateral deflection, in.-lb

$q_{u}=$ factored load per unit area, $\mathrm{lb} / \mathrm{ft}^{2}$

$Q \quad=$ stability index for a story

$r \quad=$ radius of gyration of cross section, in.

$r_{b} \quad=$ bend radius at the inside of a bar, in.

$R=$ cumulative load effect of service rain load

$s=$ center-to-center spacing of items, such as longitudinal reinforcement, transverse reinforcement, tendons, or anchors, in.

$s_{i} \quad=$ center-to-center spacing of reinforcement in the $i$-th direction adjacent to the surface of the member, in.

$s_{o}=$ center-to-center spacing of transverse reinforcement within the length $\ell_{o}$, in.

$s_{s} \quad=$ sample standard deviation, $\mathrm{psi}$

$s_{w}=$ clear distance between adjacent webs, in.

$s_{2}=$ center-to-center spacing of longitudinal shear or torsional reinforcement, in.

$S=$ effect of service snow load

$S_{D S}=5$ percent damped, spectral response acceleration parameter at short periods determined in accordance with the general building code

$S_{e}=$ moment, shear, or axial force at connection corresponding to development of probable strength at intended yield locations, based on the governing mechanism of inelastic lateral deformation, considering both gravity and earthquake effects

$S_{m}=$ elastic section modulus, in. ${ }^{3}$

$S_{n}=$ nominal moment, shear, axial, torsion, or bearing strength

$S_{y} \quad=$ yield strength of connection, based on $f_{y}$ of the connected part, for moment, shear, torsion, or axial force, psi

$t=$ wall thickness of hollow section, in.

$t_{f}=$ thickness of flange, in.

$t_{s l}=$ thickness of shear lug, in.

$T=$ cumulative effects of service temperature, creep, shrinkage, differential settlement, and shrinkagecompensating concrete

$T_{c r}=$ cracking torsional moment, in.-lb

$T_{t} \quad=$ total test load, $\mathrm{lb}$

$T_{t h}=$ threshold torsional moment, in.-lb

$T_{n}=$ nominal torsional moment strength, in.-lb

$T_{u}=$ factored torsional moment at section, in. $-1 \mathrm{~b}$

$U=$ strength of a member or cross section required to resist factored loads or related internal moments and forces in such combinations as stipulated in this Code

$v_{c}=$ stress corresponding to nominal two-way shear strength provided by concrete, psi

\section*{COMMENTARY}
$R=$ reaction, $\mathrm{lb}$

\section*{CODE}
$v_{n}=$ equivalent concrete stress corresponding to nominal two-way shear strength of slab or footing, psi

$v_{s}=$ equivalent concrete stress corresponding to nominal two-way shear strength provided by reinforcement, psi

$v_{u}=$ maximum factored two-way shear stress calculated around the perimeter of a given critical section, psi

$v_{u v}=$ factored shear stress on the slab critical section for two-way action, from the controlling load combination, without moment transfer, psi

$V_{b}=$ basic concrete breakout strength in shear of a single anchor in cracked concrete, $\mathrm{lb}$

$V_{b r g, s l}=$ nominal bearing strength of a shear lug in direction of shear, $\mathrm{lb}$

$V_{c}=$ nominal shear strength provided by concrete, $\mathrm{lb}$

$V_{c b}=$ nominal concrete breakout strength in shear of a single anchor, $\mathrm{lb}$

$V_{c b g}=$ nominal concrete breakout strength in shear of a group of anchors, $\mathrm{lb}$

$V_{c b, s l}=$ nominal concrete breakout strength in shear of attachment with shear lugs, $\mathrm{lb}$

$V_{c i}=$ nominal shear strength provided by concrete where diagonal cracking results from combined shear and moment, $\mathrm{lb}$

$V_{c p}=$ nominal concrete pryout strength of a single anchor, $\mathrm{lb}$

$V_{c p g}=$ nominal concrete pryout strength of a group of anchors, $\mathrm{lb}$

$V_{c w}=$ nominal shear strength provided by concrete where diagonal cracking results from high principal tensile stress in web, $1 \mathrm{~b}$

$V_{d}=$ shear force at section due to unfactored dead load, $\mathrm{lb}$

$V_{e}=$ design shear force for load combinations including earthquake effects, $1 \mathrm{~b}$

$V_{i}=$ factored shear force at section due to externally applied loads occurring simultaneously with $M_{\max }$, $\mathrm{lb}$

$V_{n}=$ nominal shear strength, $\mathrm{lb}$

$V_{n h}=$ nominal horizontal shear strength, $\mathrm{lb}$

$V_{p}=$ vertical component of effective prestress force at section, $\mathrm{lb}$

$V_{s}=$ nominal shear strength provided by shear reinforcement, $\mathrm{lb}$

$V_{s a}=$ nominal shear strength of a single anchor or individual anchor in a group of anchors as governed by the steel strength, $\mathrm{lb}$

$V_{u}=$ factored shear force at section, $\mathrm{lb}$

$V_{u a}=$ factored shear force applied to a single anchor or group of anchors, lb

\section*{CODE}
$V_{u a, g}=$ total factored shear force applied to anchor group, $\mathrm{lb}$

$V_{u a, i}=$ factored shear force applied to most highly stressed anchor in a group of anchors, $\mathrm{lb}$

$V_{u h}=$ factored shear force along contact surface in composite concrete flexural member, $\mathrm{lb}$

$V_{u s}=$ factored horizontal shear in a story, $\mathrm{lb}$

$V_{u, x}=$ factored shear force at section in the x-direction, $\mathrm{lb}$

$V_{u, y}=$ factored shear force at section in the y-direction, $\mathrm{lb}$

$V_{n, x}=$ shear strength in the x-direction

$V_{n, y}=$ shear strength in the y-direction

$w_{c}=$ density, unit weight, of normalweight concrete or equilibrium density of lightweight concrete, $\mathrm{lb} / \mathrm{ft}^{3}$

$\mid w_{t} \quad=$ effective tie width in a strut-and-tie model, in.

$w_{u}=$ factored load per unit length of beam or one-way slab, lb/in.

$w / c m=$ water-cementitious materials ratio

$W \quad=$ effect of wind load

$y_{t}=$ distance from centroidal axis of gross section, neglecting reinforcement, to tension face, in.

$\alpha=$ angle defining the orientation of reinforcement

$\alpha_{c}=$ coefficient defining the relative contribution of concrete strength to nominal wall shear strength

$\alpha_{f}=$ ratio of flexural stiffness of beam section to flexural stiffness of a width of slab bounded laterally by centerlines of adjacent panels, if any, on each side of the beam

$\alpha_{f n}=$ average value of $\alpha_{f}$ for all beams on edges of a panel

$\alpha_{s}=$ constant used to calculate $V_{c}$ in slabs and footings

$\alpha_{1}=$ minimum angle between unidirectional distributed reinforcement and a strut

$\beta=$ ratio of long to short dimensions: clear spans for two-way slabs, sides of column, concentrated load or reaction area; or sides of a footing

$\beta_{b} \quad=$ ratio of area of reinforcement cut off to total area of tension reinforcement at section

$\beta_{c}=$ confinement modification factor for struts and nodes in a strut-and-tie model

$\beta_{d n s}=$ ratio used to account for reduction of stiffness of columns due to sustained axial loads

$\beta_{d s}=$ the ratio of maximum factored sustained shear within a story to the maximum factored shear in that story associated with the same load combination

$\beta_{n} \quad=$ factor used to account for the effect of the anchorage of ties on the effective compressive strength of a nodal zone

$\beta_{s}=$ factor used to account for the effect of cracking and confining reinforcement on the effective compressive strength of the concrete in a strut

\section*{COMMENTARY}
$w_{s}=$ width of a strut perpendicular to the axis of the strut, in.

$w_{t}=$ effective height of concrete concentric with a tie, used to dimension nodal zone, in.

$w_{t, \text { max }}=$ maximum effective height of concrete concentric with a tie, in.

$$
W_{a}=\text { service-level wind load, } \mathrm{lb}
$$

$$
\mid \alpha_{f}=E_{c b} I_{b} / E_{c s} I_{s}
$$

\section*{CODE}
$\beta_{1}=$ factor relating depth of equivalent rectangular compressive stress block to depth of neutral axis

$\gamma_{f}=$ factor used to determine the fraction of $M_{s c}$ transferred by slab flexure at slab-column connections

$\gamma_{p} \quad=$ factor used for type of prestressing reinforcement

$\gamma_{s}=$ factor used to determine the portion of reinforcement located in center band of footing

$\gamma_{v}=$ factor used to determine the fraction of $M_{s c}$ transferred by eccentricity of shear at slab-column connections

$\delta \quad=$ moment magnification factor used to reflect effects of member curvature between ends of a compression member

$\delta_{c}=$ wall displacement capacity at top of wall, in.

$\delta_{s}=$ moment magnification factor used for frames not braced against sidesway, to reflect lateral drift resulting from lateral and gravity loads

$\delta_{u}=$ design displacement, in.

$\Delta_{c r}=$ calculated out-of-plane deflection at midheight of wall corresponding to cracking moment $M_{c r}$, in.

$\Delta_{n}=$ calculated out-of-plane deflection at midheight of wall corresponding to nominal flexural strength $M_{n}$, in.

$\Delta_{o}=$ relative lateral deflection between the top and bottom of a story due to $V_{u s}$, in.

$\Delta f_{p}=$ increase in stress in prestressed reinforcement due to factored loads, psi

$\Delta f_{p s}=$ stress in prestressed reinforcement at service loads less decompression stress, psi

$\Delta_{r} \quad=$ residual deflection measured 24 hours after removal of the test load. For the first load test, residual deflection is measured relative to the position of the structure at the beginning of the first load test. For the second load test, residual deflection is measured relative to the position of the structure at the beginning of the second load test, in.

$\Delta_{s}=$ out-of-plane deflection due to service loads, in.

$\Delta_{u}=$ calculated out-of-plane deflection at midheight of wall due to factored loads, in.

$\Delta_{x}=$ design story drift of story $x$, in.

$\Delta_{1}=$ maximum deflection, during first load test, measured 24 hours after application of the full test load, in.

$\Delta_{2}=$ maximum deflection, during second load test, measured 24 hours after application of the full test load. Deflection is measured relative to the position of the structure at the beginning of the second load test, in.

\section*{COMMENTARY}
\section*{CODE}
$\varepsilon_{t}=$ net tensile strain in extreme layer of longitudinal tension reinforcement at nominal strength, excluding strains due to effective prestress, creep, shrinkage, and temperature

$\varepsilon_{t y}=$ value of net tensile strain in the extreme layer of longitudinal tension reinforcement used to define a compression-controlled section

$\theta=$ angle between axis of strut, compression diagonal, or compression field and the tension chord of the members

$\lambda=$ modification factor to reflect the reduced mechanical properties of lightweight concrete relative to normalweight concrete of the same compressive strength

$\lambda_{a}=$ modification factor to reflect the reduced mechanical properties of lightweight concrete in certain concrete anchorage applications

$\lambda_{\Delta}=$ multiplier used for additional deflection due to long-term effects

$\lambda_{s}=$ factor used to modify shear strength based on the effects of member depth, commonly referred to as the size effect factor.

$=$ coefficient of friction

$\xi=$ time-dependent factor for sustained load

$\rho \quad=$ ratio of $A_{s}$ to $b d$

$\rho^{\prime} \quad=$ ratio of $A_{s}{ }^{\prime}$ to $b d$

$\rho_{\ell}=$ ratio of area of distributed longitudinal reinforcement to gross concrete area perpendicular to that reinforcement

$\rho_{p} \quad=$ ratio of $A_{p s}$ to $b d_{p}$

$\rho_{s}=$ ratio of volume of spiral reinforcement to total volume of core confined by the spiral, measured out-to-out of spirals

$\rho_{t}=$ ratio of area of distributed transverse reinforcement to gross concrete area perpendicular to that reinforcement

$\rho_{v} \quad=$ ratio of tie reinforcement area to area of contact surface

$\rho_{w} \quad=$ ratio of $A_{s}$ to $b_{w} d$

$\phi \quad=$ strength reduction factor

$\phi_{p}=$ strength reduction factor for moment in pretensioned member at cross section closest to the end of the member where all strands are fully developed

$\sigma=$ wall boundary extreme fiber concrete nominal compressive stress, psi

\section*{COMMENTARY}
 erties is caused by the reduced ratio of tensileto-compressive strength of lightweight concrete compared to normalweight concrete. There are instances in the Code where $\lambda$ is used as a modifier to reduce expected performance of lightweight concrete where the reduction is not related directly to tensile strength. $\lambda=$ in most cases, the reduction in mechanical prop-\includegraphics[max width=\textwidth, center]{2024_04_28_4d3e034aecfbb1c9f4b3g-15}\\
$\varsigma \quad=$ exponent symbol in tensile/shear force interaction equation\\
$\phi_{K} \quad=$ stiffness reduction factor

$\tau_{c r}=$ characteristic bond stress of adhesive anchor in cracked concrete, psi

\section*{CODE}
$\tau_{\text {uncr }}=$ characteristic bond stress of adhesive anchor in uncracked concrete, psi

$\psi_{\text {brg } s l}=$ shear lug bearing factor used to modify bearing strength of shear lugs based on the influence of axial load

$\psi_{c}=$ factor used to modify development length based on concrete strength

$\psi_{c, N}=$ breakout cracking factor used to modify tensile strength of anchors based on the influence of cracks in concrete

$\psi_{c, P}=$ pullout cracking factor used to modify pullout strength of anchors based on the influence of cracks in concrete

$\psi_{c, V}=$ breakout cracking factor used to modify shear strength of anchors based on the influence of cracks in concrete and presence or absence of supplementary reinforcement

$\psi_{c p, N}=$ breakout splitting factor used to modify tensile strength of post-installed anchors intended for use in uncracked concrete without supplementary reinforcement to account for the splitting tensile stresses

$\psi_{c p, N a}=$ bond splitting factor used to modify tensile strength of adhesive anchors intended for use in uncracked concrete without supplementary reinforcement to account for the splitting tensile stresses due to installation

$\psi_{e}=$ factor used to modify development length based on reinforcement coating

$\psi_{e c, N}=$ breakout eccentricity factor used to modify tensile strength of anchors based on eccentricity of applied loads

$\psi_{e c, N a}=$ breakout eccentricity factor used to modify tensile strength of adhesive anchors based on eccentricity of applied loads

$\psi_{e c, V}=$ breakout eccentricity factor used to modify shear strength of anchors based on eccentricity of applied loads

$\psi_{e d, N}=$ breakout edge effect factor used to modify tensile strength of anchors based on proximity to edges of concrete member

$\psi_{e d, N a}=$ breakout edge effect factor used to modify tensile strength of adhesive anchors based on proximity to edges of concrete member

$\psi_{e d, V}=$ breakout edge effect factor used to modify shear strength of anchors based on proximity to edges of concrete member

$\psi_{g}=$ factor used to modify development length based on grade of reinforcement

$\psi_{h, V}=$ breakout thickness factor used to modify shear strength of anchors located in concrete members with $h_{a}<1.5 c_{a 1}$

$\psi_{o} \quad=$ factor used to modify development length of hooked and headed bars based on side cover and confinement

\section*{CODE}
$\psi_{p}=$ factor used to modify development length for headed reinforcement based on parallel tie reinforcement

$\psi_{r}=$ factor used to modify development length based on confining reinforcement

$\psi_{s}=$ factor used to modify development length based on reinforcement size

$\psi_{t}=$ factor used to modify development length for casting location in tension

$\psi_{w}=$ factor used to modify development length for welded deformed wire reinforcement in tension

$\Omega_{o}=$ amplification factor to account for overstrength of the seismic-force-resisting system determined in accordance with the general building code

$\Omega_{v} \quad=$ overstrength factor equal to the ratio of $M_{p r} / M_{u}$ at the wall critical section

$\omega_{v} \quad=$ factor to account for dynamic shear amplification

\section*{2.3-Terminology}
adhesive - chemical components formulated from organic polymers, or a combination of organic polymers and inorganic materials that cure if blended together.

admixture - material other than water, aggregate, cementitious materials, and fiber reinforcement used as an ingredient, which is added to grout, mortar, or concrete, either before or during its mixing, to modify the freshly mixed, setting, or hardened properties of the mixture.

aggregate - granular material, such as sand, gravel, crushed stone, iron blast-furnace slag, or recycled aggregates including crushed hydraulic cement concrete, used with a cementing medium to form concrete or mortar.

aggregate, lightweight-aggregate meeting the requirements of ASTM C330 and having a loose bulk density of $70 \mathrm{lb} / \mathrm{ft}^{3}$ or less, determined in accordance with ASTM C29.

alternative cement - an inorganic cement that can be used as a complete replacement for portland cement or blended hydraulic cement, and that is not covered by applicable specifications for portland or blended hydraulic cements.

anchor - a steel element either cast into concrete or post-installed into a hardened concrete member and used to transmit applied loads to the concrete.

\section*{COMMENTARY}
\section*{R2.3-Terminology}
aggregate-The use of recycled aggregate is addressed in the Code in 2019. The definition of recycled materials in ASTM C33 is very broad and is likely to include materials that would not be expected to meet the intent of the provisions of this Code for use in structural concrete. Use of recycled aggregates including crushed hydraulic-cement concrete in structural concrete requires additional precautions. See 26.4.1.2.1(c).

aggregate, lightweight - In some standards, the term "lightweight aggregate" is being replaced by the term "lowdensity aggregate."

alternative cements-Alternative cements are described in the references listed in R26.4.1.1.1(b). Refer to 26.4.1.1.1(b) for precautions when using these materials in concrete covered by this Code.

anchor - Cast-in anchors include headed bolts, hooked bolts (J- or L-bolt), and headed studs. Post-installed anchors include expansion anchors, undercut anchors, screw anchors, and adhesive anchors; steel elements for adhesive anchors include threaded rods, deformed reinforcing bars, or internally threaded steel sleeves with external deformations. Anchor types are shown in Fig. R2.1.


\end{document}